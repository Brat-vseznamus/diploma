\documentclass[11pt]{article}

\usepackage{sectsty}
\usepackage{graphicx}
\usepackage{mathtools}

% Margins
\topmargin=-0.45in
\evensidemargin=0in
\oddsidemargin=0in
\textwidth=6.5in
\textheight=9.0in
\headsep=0.25in

\title{ Translation }
% \author{ Author }
% \date{\today}

\newcommand{\ga}{\alpha}
\newcommand{\gb}{\beta}
\newcommand{\gl}{\lambda}
\newcommand{\gt}{\tau}



\begin{document}
\maketitle	

\section*{}


As is well known, in the type-free $\lambda$-calculus exactly the recursive functions are
$definable^1$. The concept used here of the definability of a function
is also useful for the $\lambda$-calculus with types. We answer the question $here^2$ according to the then definable functions.

Types are $0$ and with $\sigma$, $\tau$ also \((\sigma \rightarrow \tau)\). Terms (denoted by \(r^\tau\), \(s^\tau\), \(t^\tau\))
formed from variables with types by application and \(\lambda\)-abstraction. Type indices that result from the context or that are immaterial,
we often leave it out. Terms that differ only by bound renaming
divorce, are identified. The relation \(t \models t'\) ($t$ is reducible to $t'$) becomes inductively defined by

\begin{enumerate}
    \item $(\lambda x. t) s \models t_x[s]$
    \item When \(t \models t'\) and \(s \models s'\), so \(t s \models t' s'\)
    \item When \(t \models t'\), so \(\lambda x.t \models \lambda x.t'\)
    \item \(t \models t\)
    \item When \(t \models t'\) and \(t' \models t''\), so \(t \models t''\)
\end{enumerate}

A term is called in normal form if it has no subterms of the form \((\lambda x.t) s\). As is well known, for every term $t$ there is a uniquely determined normal form \(t'\) with \(t \models t'\). Two terms are equal if they have the same normal form. One can now introduce natural numbers as terms 
\( \overline{n} \equiv \lambda \alpha . \alpha^n\) vom
type \(\nu \equiv (0 \rightarrow 0) \rightarrow (0 \rightarrow 0)\); where \(\alpha^n\) is the n-th iterate of \(\alpha\), i.e. \(\equiv \lambda x.\alpha(\alpha...(\alpha x)...)\).
Every closed term of type \(\nu\) in normal form is an \(\overline{n}\). So defined every closed term \(t\) of the type \(\nu \rightarrow (\nu \rightarrow ...(\nu \rightarrow \nu)...)\) is a number-theoretic one
Function $f$ determined by $t\overline{n_1},...\overline{n_k} = \overline{f(n_1,...,n_k)}$. For example $will^3$

\begin{enumerate}
    \item \(n + m\) define as \(\lambda F G \alpha. (F \alpha) \circ (G \alpha)\)
    \item \(n \cdot m\) define as \(\lambda F G \alpha. F \circ G\)
    \item \(k\) (constant function) define as \(\lambda F \alpha . \alpha ^ k\)
    \item \[ d(n, m, i) =
        \begin{dcases}
            n & i = 0 \\
            m & i \neq 0 \\
        \end{dcases}
    \] define as \(\lambda F G H \alpha x. H(\lambda y. G \alpha x) (F \alpha x)\)
\end{enumerate}

[F, G, H variables of type \(\nu, t^{\tau \rightarrow \tau} \circ s^{\tau \rightarrow \tau} \equiv \lambda x^\tau. t(s x)\)]. The set of
bare functions is apparently closed against insertions. So are all Polynomials definable, and more generally all functions resulting from polynomials Case distinction according to disappearance or non-disappearance of arguments
are definable; in the 2-digit case, these are the functions

\[ f(n, m) =
        \begin{dcases}
            k & n = 0 \text{ and } m = 0 \\
            P_1(m) & n = 0 \text{ and } m \neq 0 \\
            P_2(m) & n \neq 0 \text{ and } m = 0 \\
            P_3(n, m) & n \neq 0 \text{ and } m \neq 0 \\
        \end{dcases}
\]

with polynomials \(P_1, P_2, P_3\). In the following it will be shown that this is all definable functions are.

So \(t\) defines a roughly 2-place function. Bring \(t F G \alpha\) to normal form.

Each subterm (of the normal form of \(t F G \alpha\)) has one of the types $0$, $0 \rightarrow 0$, or \(\nu\). Each subterm of type \(\nu\) is identical to $F$ or $G$. Possible subterms from Type \(0 \rightarrow 0\) are:

\begin{enumerate}
    \item \(\alpha\)
    \item With $s$ also $Fs$, $Gs$.
    \item With \(s_1, ..., s_q\), formed according to (1), (2) also \(\lambda y. s_1 (... s_q(z)...)\), where $z$ also iden-
    table with $y$ can be.
\end{enumerate}

In the following, \(s\) stands for subterms of the type \(0 \rightarrow 0\). — For $F$, $G$ set $\overline{n}$, $\overline{m}$, where initially \(n, m \geq 1\) is assumed. By induction over s one shows: Each \(s' \equiv s_{F, G} [\overline{n}, \overline{m}]\) is equal to \(\alpha^{P(n, m)}\) or equal to a constant function \(\gl y. \alpha^{P(n, m)} z\) (\(P(n, m)\) polynom). Proof: To (2)

\begin{align*} 
    (\gl \gb. \gb^n) \ga^{P(n, m)} &= \ga^{P(n, m) \cdot n}\\ 
    (\gl \gb. \gb^n) (\gl y. \ga^{P(n, m) } z) &= \gl y. \ga^{P(n, m)} z  & (n \geq 1)
\end{align*}

To (3). Case 1: None of the \(s_i'\) is constant.

\begin{align*}
    \gl y. s_1' (...s_q'(z)...) = \gl y. \ga^{P_1(n, m) + \dots + P_q(n, m)} z
\end{align*}

Case 2: There is a first constant \(s_i'\), say \(s_i' = \gl \tilde{y}. \ga^{P_i(n, m)} \tilde{z}\)

\begin{align*}
    \gl y. s_1' (...s_q'(z)...) = \gl y. \ga^{P_1(n, m) + \dots + P_q(n, m)} z
\end{align*}

Since the normal form of \(t F G \ga\) does not contain a free variable of type $0$, it is nach replacing $F$, $G$ with \(\overline{n}\), \(\overline{m}\) $(n,m \geq 1)$ equal to an \(\ga^{P(n, m)}\).

If about \(n = 0, m \geq 1\), then one can replace all outermost subterms of the form \(F s\) by \(\gl x. x\) and gets \(\ga^{P(m)}\) like before.


\end{document}