\documentclass[11pt]{article}

\usepackage[english]{babel}
\usepackage{amsthm}
\usepackage{sectsty}
\usepackage{graphicx}
\usepackage{mathtools}

% Margins
\topmargin=-0.45in
\evensidemargin=0in
\oddsidemargin=0in
\textwidth=6.5in
\textheight=9.0in
\headsep=0.25in

\title{ Proof }
% \author{ Author }
% \date{\today}

\newtheorem{theorem}{Theorem}[section]
\newtheorem{corollary}{Corollary}[theorem]
\newtheorem{lemma}[theorem]{Lemma}

\newcommand{\ga}{\alpha}
\newcommand{\gb}{\beta}
\newcommand{\gl}{\lambda}
\newcommand{\gt}{\tau}
\newcommand{\too}{\rightarrow}


\begin{document}
\maketitle	

\section*{}

\begin{align*}
    \nu = (0 \too 0) \too (0 \too 0) \\
\end{align*}

Let's consider \(t\) - two-digit function and \(N, M\) - two numerals and \(f\) as general number function, then their types are:

\begin{align*}
    t &: \nu \too \nu \too \nu \\
    N, M &: \nu \\
    f &: 0 \too 0
\end{align*}

Then lets consider normal form of \(t' = t N M f : 0 \too 0\)

Import to notice that \(t\) is closed.

We will proof that any subterm of \(t'\) has one of types from set \(S = \{0, 0 \too 0, \nu\}\).

Declare function for our comfort in usage \(T\) which return type of expression in some context: 

\begin{align*}
    \forall e - \text{term}, \Gamma - \text{context} : T(\Gamma, e) = \gt,\ \text{if \(\Gamma \vdash e:\gt\)}
\end{align*}

For this let introduce Lemma which will helps us in it:

\begin{lemma}
    Consider some expression in normal form \(e\) in context \(\Gamma\) with such requirements:
    \begin{enumerate}
        \item Lets consider \(e\) has type \(\tau\) then \(\tau\) must be in set \(S\)
        \item For each free variable \(x \in FV(e)\) is true that \(T(\Gamma, x) \in S\)
    \end{enumerate}
    Then each subterm of \(e\) also has type which is in set \(S\)
\end{lemma}

\begin{proof}
    
    We will proof this by induction on each step considering expression constructed by one of the possible ways.

    Consider all possible deconstructions of expression \(e\)

    \begin{enumerate}
        \item \(e = c_i\) where \(c_i \in FV(e)\)
        \item \(e = \gl x. B\) then \(\tau = t_1 \too t_2\)
        \item \(e = P Q\) then \(\exists \ga. P : \ga \too \tau, Q : \ga\)
    \end{enumerate}

    For case 1 the lemma is explicitly satisfied since \(e\) doesn't have any subterm, except itself so lemma is correct

    For case 2 consider few options
    
    Option 1:
    \begin{align*}
        \begin{dcases}
            \tau &= 0 \too 0 \\
            t_1 &= 0 \\
            t_2 &= 0
        \end{dcases}
    \end{align*}

    Then \(FV(B) = FB(e) \cup \{x\}\) and because of B exists in context \(\Gamma' = \Gamma, x:0\) and has type \(0\) then by induction case is clear that each \(B\)'s subterm has correct type, but because if \(x\) s is subterm \(B\), then \(x\) is subterm of \(e\).

    Other option:
    \begin{align*}
        \begin{dcases}
            \tau &= \nu \\
            t_1 &= 0 \too 0 \\
            t_2 &= 0 \too 0
        \end{dcases}
    \end{align*}

    Same proof as for option 1.

    For case 3 let's recall that \(e\) in normal form that means \(P\) - is not lambda (otherwise it would be reduced) and two options

    \begin{enumerate}
        \item \(P \in FV(e)\) than the same variants for type \(\tau\) as for case 2 with induction step for \(Q\)
        \item \(P = P_2 Q_1\)

    \end{enumerate}

    We can repeat choosing option 2 finite number of times because we are in normal form.

    Then consider

    \begin{align*}
        e = P A_1 A_2 ... A_k, \text{where \(P \in FV(e)\)}
    \end{align*}

    Again few options:

    \begin{align*}
        \begin{dcases}
            e &: 0 \\
            P &: \nu \\
            A_1 &: 0 \to 0 \\
            A_2 &: 0
        \end{dcases}
    \end{align*}

    \begin{align*}
        \begin{dcases}
            e &: 0 \\
            P &: 0 \to 0 \\
            A_1 &: 0 \\
        \end{dcases}
    \end{align*}

    \begin{align*}
        \begin{dcases}
            e &: 0 \to 0 \\
            P &: \nu \\
            A_1 &: 0 \to 0 \\
        \end{dcases}
    \end{align*}

    Again induction

\end{proof}

\end{document}